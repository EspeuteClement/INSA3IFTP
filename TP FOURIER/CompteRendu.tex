\documentclass[12pt]{article}

\usepackage{custom}

\title{Compte-Rendu TP Fourier}
\author{Pair B3408 \\ {\sc Renault} Benoit, {\sc Espeute} Clément}
\date{}
\usepackage[top=1in, bottom=1.25in, left=1.25in, right=1.25in]{geometry}
\usepackage{listings}
\usepackage{graphicx}
\usepackage{hyperref}

\definecolor{commentColor}{rgb}{0.7,0.7,0.7}
\definecolor{mygray}{rgb}{0.5,0.5,0.5}
\definecolor{mymauve}{rgb}{0.58,0,0.82}
\lstset{language=bash} 
\lstset{ %
  backgroundcolor=\color{black!5},   % choose the background color; you must add \usepackage{color} or \usepackage{xcolor}
  basicstyle=\footnotesize\ttfamily,        % the size of the fonts that are used for the code
  breakatwhitespace=false,         % sets if automatic breaks should only happen at whitespace
  breaklines=true,                 % sets automatic line breaking
  captionpos=b,                    % sets the caption-position to bottom
  commentstyle=\color{commentColor},    % comment style
  deletekeywords={...},            % if you want to delete keywords from the given language
  escapeinside={\%*}{*)},          % if you want to add LaTeX within your code
  extendedchars=true,              % lets you use non-ASCII characters; for 8-bits encodings only, does not work with UTF-8
  frame=single,	                   % adds a frame around the code
  keepspaces=true,                 % keeps spaces in text, useful for keeping indentation of code (possibly needs columns=flexible)
  keywordstyle=\color{blue},       % keyword style
  language=C++,                 % the language of the code
  otherkeywords={*,...},            % if you want to add more keywords to the set
  numbers=left,                    % where to put the line-numbers; possible values are (none, left, right)
  numbersep=5pt,                   % how far the line-numbers are from the code
  numberstyle=\tiny\color{mygray}, % the style that is used for the line-numbers
  rulecolor=\color{black!20},         % if not set, the frame-color may be changed on line-breaks within not-black text (e.g. comments (green here))
  showspaces=false,                % show spaces everywhere adding particular underscores; it overrides 'showstringspaces'
  showstringspaces=false,          % underline spaces within strings only
  showtabs=false,                  % show tabs within strings adding particular underscores
  stepnumber=1,                    % the step between two line-numbers. If it's 1, each line will be numbered
  stringstyle=\color{mymauve},     % string literal style
  tabsize=2,	                   % sets default tabsize to 2 spaces
  title=\lstname,                   % show the filename of files included with \lstinputlisting; also try caption instead of title
  postbreak=\raisebox{0ex}[0ex][0ex]{\tt\color{red}-> }
}

\begin{document}
\pagestyle{fancy}
\maketitle

\section{Transformée de Fourier discrète}
\subsection{Quelle est la période d'échantillonnage $T_e$ ?}
La période d'échantillonage est de $T_e = \frac{b-a}{N} = \frac{5+5}{32768} \approx{0,00030517578125}$ secondes.

\subsection{Quelle est la fréquence d'échantillonnage $f_e$ ?}
La période d'échantillonage est de $f_e = \frac{N}{b-a} = \frac{32768}{10} = 3276,8$ hertzs.

\subsection{Affichage des fonctions et leurs spectres :}
Cf. : Exécuter les programmes MatLab.

\subsection{Comment choisir $f_0$ pour avoir un nombre entier de périodes entre a et b ?}
Il suffit de choisir $f_0$ tel que $Nb_\text{périodes} = \frac{f_0}{b-a}\in Z$.

\subsection{Et si on choisit $f_0$ pour que ça ne soit pas le cas ?}
On obtient des décompositions complètement à côté de la plaque (Cf. Screenshots). Cela s'explique par le fait que les fonctions ne sont alors plus périodiques.

\subsection{Version périodique de x6(t) :}
La version périodique donne des transformées ressemblant à des sinus cardinaux "échantillonés".

\subsection{Calcul de la transformée de Fourier de $x_7(t)$ :}
$$X_7(\omega) = \int_{-\infty}^{+\infty} x_7(t) e^{-i\omega t}dt$$

$$= \int_{-\infty}^{+\infty} e^{-\pi t^2} e^{-i\omega t}dt$$

$$= \int_{-\infty}^{+\infty} e^{-\pi t^2 -i\omega t}dt$$

$$= \int_{-\infty}^{+\infty} e^{-((\sqrt{\pi} t)^2 +i\omega t + (\frac{i\omega}{2})^2) + (\frac{i\omega}{2\sqrt{\pi}})^2}dt$$

$$= e^{(\frac{i\omega}{2\sqrt{\pi}})^2} \int_{-\infty}^{+\infty} e^{-(\sqrt{\pi}t + \frac{i\omega}{2\sqrt{\pi}})^2}dt$$

$$= \frac{e^{(\frac{i\omega}{2\sqrt{\pi}})^2}}{\sqrt{\pi}} \int_{-\infty}^{+\infty} e^{-x^2}dx$$

$$= e^{(\frac{i\pi f}{\sqrt{\pi}})^2}$$

$$= e^{-\pi f^2}$$

\section{Échantillonnage et aliasing}
\subsection{Spectres théoriques des fonctions $g_{800}$ et $g_{2400}$}
On s'attend à trouver des diracs à $\pm \{800, 840, 880\}$ dans les parties réelles et imaginaires ainsi que le module, et de même pour $\pm \{2400, 2440, 2480\}$.

\subsection{Différences avec les simulations}
On retrouve bien les diracs prévus pour $g_{800}$, mais pas pour $g_{2400}$, car dans ce second cas, le nombre d'échantillons est trop faible pour assurer un échantillonage correct de la fonction.

\subsection{Simple sinus}
On obtient un simple sinus (un dirac symétrique) en zéro pour les $f=f_e-40$ et aux extrémités du graphe pour $f=\frac{f_e}{2}-40$.

\end{document}