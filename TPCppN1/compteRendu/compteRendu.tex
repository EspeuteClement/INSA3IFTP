\documentclass[11pt]{article}

\usepackage{custom}

\title{Compte-Rendu TP C++ \#1}
\author{{\sc Renault} Benoit, {\sc Espeute} Clément}
\date{14 septembre 2015}

\usepackage{listings}


\definecolor{commentColor}{rgb}{0.7,0.7,0.7}
\definecolor{mygray}{rgb}{0.5,0.5,0.5}
\definecolor{mymauve}{rgb}{0.58,0,0.82}
\lstset{language=bash} 
\lstset{ %
  backgroundcolor=\color{black!5},   % choose the background color; you must add \usepackage{color} or \usepackage{xcolor}
  basicstyle=\footnotesize\ttfamily,        % the size of the fonts that are used for the code
  breakatwhitespace=false,         % sets if automatic breaks should only happen at whitespace
  breaklines=true,                 % sets automatic line breaking
  captionpos=b,                    % sets the caption-position to bottom
  commentstyle=\color{commentColor},    % comment style
  deletekeywords={...},            % if you want to delete keywords from the given language
  escapeinside={\%*}{*)},          % if you want to add LaTeX within your code
  extendedchars=true,              % lets you use non-ASCII characters; for 8-bits encodings only, does not work with UTF-8
  frame=single,	                   % adds a frame around the code
  keepspaces=true,                 % keeps spaces in text, useful for keeping indentation of code (possibly needs columns=flexible)
  keywordstyle=\color{blue},       % keyword style
  language=C++,                 % the language of the code
  otherkeywords={*,...},            % if you want to add more keywords to the set
  numbers=left,                    % where to put the line-numbers; possible values are (none, left, right)
  numbersep=5pt,                   % how far the line-numbers are from the code
  numberstyle=\tiny\color{mygray}, % the style that is used for the line-numbers
  rulecolor=\color{black!20},         % if not set, the frame-color may be changed on line-breaks within not-black text (e.g. comments (green here))
  showspaces=false,                % show spaces everywhere adding particular underscores; it overrides 'showstringspaces'
  showstringspaces=false,          % underline spaces within strings only
  showtabs=false,                  % show tabs within strings adding particular underscores
  stepnumber=1,                    % the step between two line-numbers. If it's 1, each line will be numbered
  stringstyle=\color{mymauve},     % string literal style
  tabsize=2,	                   % sets default tabsize to 2 spaces
  title=\lstname,                   % show the filename of files included with \lstinputlisting; also try caption instead of title
  postbreak=\raisebox{0ex}[0ex][0ex]{\tt\color{red}-> }
}


\begin{document}
\pagestyle{fancy}
\maketitle

\section{Spécifications de la classe <Collection>}

\subsection{Présentation générale}
<Collection> a pour rôle de gérer le stockage, la modification et l'affichage d'un ensemble de \emph{nombres entiers}. Le stockage est réalisé dans un tableau d'entiers (int) géré de façon dynamique simple, c'est à dire que l'on a également deux attributs de type entier non-signé, l'un servant à connaître le nombre d'éléments actuellement présents dans le tableau, l'autre servant à connaître la taille allouée au tableau. Le cahier des charges ne l'exigeant pas, les redondances sont acceptées et les valeurs ne sont pas triées.

\subsection{Présentation des méthodes}

\subsubsection*{Constructeur n°1}
Créé une collection pouvant accueillir un nombre entier non-signé donné d'éléments sans avoir besoin d'être redimensionnée, et alloue la mémoire nécessaire au tableau.

\subsubsection*{Constructeur n°2}
Créé une collection pouvant accueillir un nombre entier non-signé donné d'éléments sans avoir besoin d'être redimensionnée, et alloue la mémoire nécessaire au tableau. Les valeurs du tableau d'entiers quelconque dont le pointeur est passé en paramètre sont ensuite ajoutées à la collection (part du contrat : il est à la charge de l’utilisateur de donner une taille égale à celle du tableau passé).

\subsubsection*{Méthode Afficher}
Imprime sur la sortie standard les éléments de la collection dans l'ordre d’apparition du tableau, avec un retour chariot entre chaque élément sur la sortie standard. Si en mode de déboguage, affiche en plus le nombre d’éléments et la taille allouée à la fin de l’affichage des éléments.

\subsubsection*{Méthode Ajouter}
Ajoute la valeur entière donnée dans un élément inséré à l'extrêmité du tableau et fait appel à la fonction d'ajustement si l'insertion implique un dépassement de la taille actuellement allouée au tableau.
\footnote{Tout ajout ou suppression du nombre d'éléments contenus dans le tableau implique bien sûr une mise à jour de l'attribut associé.}

\subsubsection*{Méthode Retirer}
Supprime un nombre entier d'occurences donné d'une valeur entière donnée. Si le nombre d'occurences à retirer donné est négatif, on supprime toutes les occurrences de la valeur entière donnée. La taille du tableau est ajustée après l'opération. 

\subsubsection*{Méthode Ajuster}
Modifie la taille allouee au tableau et le réalloue dynamiquement à une valeur entière non-signée donnée en paramètre. Le redimensionnement est refusé si la taille demandée est inférieur au nombre actuel d'éléments et renvoie un code de retour \texttt{ERR\_TAILLE}.
    
\subsubsection*{Méthode Reunir}
Ajoute les éléments de la collection donnée après ceux de la courante en réajustant la taille allouee du tableau de la courante si celle-ci n'est actuellement pas assez grande pour les accueillir.

\subsubsection*{Méthode Destructeur}
Détruit la collection courante et libère la mémoire allouée au tableau associé.

\section{Les tests}
Afin de vérifier que chaque méthode  notre classe fonctionne, on réalise des tests unitaires sur chacune d'entre elles. La méthode Afficher à été modifiée pour pouvoir donner la taille allouée du tableau et le nombre d'éléments a l'intérieur de celui-ci. Cet affichage peut être désactivé en enlevant le \texttt{\#define DEBUG} à l'intérieur de \texttt{Collection.h}.

\subsection{Test pour \tt Retirer}
Les tests de la méthode retirer sont basés 

\end{document}