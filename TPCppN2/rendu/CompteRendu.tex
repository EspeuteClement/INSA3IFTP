\documentclass[10pt]{article}

\usepackage{custom}

\title{Report Assignment C++ \#2}
\author{Pair B3408 \\ {\sc Renault} Benoit, {\sc Espeute} Clément}
\date{}
\usepackage[top=1in, bottom=1.25in, left=1.25in, right=1.25in]{geometry}
\usepackage{listings}
\usepackage{graphicx}
\usepackage{hyperref}

\definecolor{commentColor}{rgb}{0.7,0.7,0.7}
\definecolor{mygray}{rgb}{0.5,0.5,0.5}
\definecolor{mymauve}{rgb}{0.58,0,0.82}
\lstset{language=bash} 
\lstset{ %
  backgroundcolor=\color{black!5},   % choose the background color; you must add \usepackage{color} or \usepackage{xcolor}
  basicstyle=\footnotesize\ttfamily,        % the size of the fonts that are used for the code
  breakatwhitespace=false,         % sets if automatic breaks should only happen at whitespace
  breaklines=true,                 % sets automatic line breaking
  captionpos=b,                    % sets the caption-position to bottom
  commentstyle=\color{commentColor},    % comment style
  deletekeywords={...},            % if you want to delete keywords from the given language
  escapeinside={\%*}{*)},          % if you want to add LaTeX within your code
  extendedchars=true,              % lets you use non-ASCII characters; for 8-bits encodings only, does not work with UTF-8
  frame=single,	                   % adds a frame around the code
  keepspaces=true,                 % keeps spaces in text, useful for keeping indentation of code (possibly needs columns=flexible)
  keywordstyle=\color{blue},       % keyword style
  language=C++,                 % the language of the code
  otherkeywords={*,...},            % if you want to add more keywords to the set
  numbers=left,                    % where to put the line-numbers; possible values are (none, left, right)
  numbersep=5pt,                   % how far the line-numbers are from the code
  numberstyle=\tiny\color{mygray}, % the style that is used for the line-numbers
  rulecolor=\color{black!20},         % if not set, the frame-color may be changed on line-breaks within not-black text (e.g. comments (green here))
  showspaces=false,                % show spaces everywhere adding particular underscores; it overrides 'showstringspaces'
  showstringspaces=false,          % underline spaces within strings only
  showtabs=false,                  % show tabs within strings adding particular underscores
  stepnumber=1,                    % the step between two line-numbers. If it's 1, each line will be numbered
  stringstyle=\color{mymauve},     % string literal style
  tabsize=2,	                   % sets default tabsize to 2 spaces
  title=\lstname,                   % show the filename of files included with \lstinputlisting; also try caption instead of title
  postbreak=\raisebox{0ex}[0ex][0ex]{\tt\color{red}-> }
}

\begin{document}
\pagestyle{fancy}
\maketitle
\section{<Sensor> class specifications}
\subsection{Global presentation}
Stores the ID of a given sensor and a pointer to a tri-dimensional index of Stats event counters thus allowing easy computation of statistics on a per minute, hour, day and sensor basis.

\subsection{Structures overview}
\subsubsection*{Structure <Stats>}
Stores event integer counters for each of the possible traffic states (v, j, r, n).

\paragraph{Constructor :}
Default constructor sets all array elements to zero.

\paragraph{Overload of operator+= :}
Allows adding two Stats structs by adding each array element separately.

\paragraph{Sum method :}
Adds the statistics of the struct instance together.

\subsubsection*{Structure <StatsRel>}
Stores relative statistics for each of the possible traffic states (v, j, r, n).

\paragraph{Constructor :}
Default constructor converts all the attributes of the passed in Stats struct into statistics relative to the sum of all the counters.

\paragraph{PrintStatsRel :}
Displays the attributes of the StatsRel struct one per line, rounded into percentages.

\subsection{Functions overview}

\subsubsection*{Constructor}
Default constructor sets ID to the one given and dynamically allocates the memory required for the tri-dimensional index.st

\subsubsection*{AddEvent}
Increments the suitable event state counter for a given day of week, hour and minute by one.

\subsubsection*{GetStatsByMin}
Getter for the Stats of a given time.

\subsubsection*{AddStatsByDay}
Adds all the Stats structs comprised in the given day to the Stats struct which pointer is passed in.

\subsubsection*{AddStatsByHour}
Adds all the Stats structs comprised in the given day and hour to the Stats struct which pointer is passed in.

\subsubsection*{AddStatsBySensor}
Adds all the Stats structs of the current Sensor to the Stats struct which pointer is passed.

\subsubsection*{GetDuration}
Computes and returns the probable duration of the journey through the road segment associated with the sensor. Each probable state is associated with a duration : (v->1, j->2, r->4, n->10). If no data if available, all the stats are equal to zero, therefore affecting 1 minute ("v" case).

\subsubsection*{GetID}
Getter for the sensor ID.

\subsubsection*{PrintSensorStatsRel}
Computes and displays the statistics of the current Sensor for every attribute (v, j, r, n).

\section{<IoEngine> class specifications}
\subsection{Global presentation}
Manages the inputs on the standard input stream and displays the according data. For details about the commands format or meaning, please directly refer either to the assignement or the documentation of the class.

\subsection{Functions overview}

\subsubsection*{Constructor}
Default constructor affects the BinaryTree pointer to the attribute.

\subsubsection*{ReadInput}
Reads the input from the STDIO and modify the tree if needed.

\subsubsection*{HandleADD}
Calls the necessary function thats adds the new event to the appropriate sensor and creates the sensor (and the associated node in the binary tree) if they don't exist yet.

\subsubsection*{HandleJAM\_DH}
Performs the JAM\_DH operation : computes and displays the percentage of traffic jam (corresponding to a "r" or "n" traffic state) for each hour (0 to 23) of a given day of the week, averaging from every sensor.

\subsubsection*{HandleOPT}
Performs the OPT operation : computes and displays the optimal time (which reduces the duration of the journey through the given segments) of departure and the associated minimal duration. It is assumed that every required sensor exists or the function will simply exit.

\subsubsection*{HandleSTATS\_C}
Performs the STATS\_C operation : displays the statistics for each of the traffic states attributes (v, j, r, n) of a sensor as percentages, one by line. It is assumed the sensor exists or the function will simply exit.

\subsubsection*{HandleSTATS\_D7}
Performs the STATS\_D7 operation : computes and displays the statistics of every traffic state (v, j, r, n) of a given day of the week, averaging from every sensor.

\section{Storing the Sensors}
We choosed to use a BinaryTree as our storage method for the Sensors. It allows us to have very fast indexing and insertion speed.
Here some explainations about the vocabulary that we'll use later :
\begin{description}
	\item[Node] Represents a element in our tree. Each one has 2 children and one parent. Each node contains a sensor. The Sensor's ID determines the position of the node in the tree, as the left child Sensor's ID is always lower than the one of it's parents, and the right one is higher.
	\item[Height] The height is the longest chain of node in a subtree.
	\item[Balance] The balance of a node is determined like this : $ leftSubtreeHeight - rightSubtreeHeight$. It allows to know if a subtree is balanced or not. In our tree, the node balance has to be between $-1$ and $+1$. If it's higher, we have to perform some operations in order to restore the balance.
\end{description}

\section{<Node> Class Specifications}
\subsection{Global Presentation}


Node is a class that represents a BinaryTree node. Each one contains a Sensor and 3 Node pointers : 1 for the parent node and 2 for the left and right child nodes. The nodes are sorted with the Sensor ID : Nodes that are at the left of this node will have their Sensor ID lower to this one, and Node to the left will have a Sensor ID higher to this one.
\subsection{Function overview}
\subsubsection*{Constructor}
By default, the node constructor only take a Sensor as an argument. You can also specify the parents node and the children (defaulted to NULL).
\subsection{GetLeft and GetRight}
Simple Getters for the Left and Rigth childs of this node.
\subsection{SwapSensor}
Exchange the Sensor of this node with another node. It allows to keep links between the nodes. This method is private because if an external user uses it, it could break the structure of the subtree.

\subsection{LeftRotation and RightRotation}
Left and right rotations are operations that doesn't change the logic of the binary tree. It's used in the rebalancing process. More explanations can be found here \url{https://en.wikipedia.org/wiki/Tree_rotation}.

This function is not as optimized as it should because I use 5 operations instead of the usual 3. This is because I use a sensor swap so we can keep the tree structure in place. If we didn't do that, some parent-child relations from node not involved in the operations could be messed up.

\subsection{DoubleLeftRotation and DoubleRightRotation}
Each one performs 2 rotations on the tree. Again, it's used in the rebalancing process of the tree.

\subsection{ComputeHeight}
This method compute the height of the current node if needed.






\end{document}